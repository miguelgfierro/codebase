%%%%%%%%%%%%%%%%%%%%%%%%%%%%%%%%%%%%%%%%%%%%%%%%%%%%%%%%%%%%%%%%%%%%%%%%%%%%%%%%
%2345678901234567890123456789012345678901234567890123456789012345678901234567890
%        1         2         3         4         5         6         7         8
\chapter{Imitation Learning and Skill Innovation in Humanoids through Reward Templates}\label{ch_imitation}
\textit{The imitation of a skill and its improvement by innovating new solutions are  the key steps to human learning. The process of human learning  has been widely studied in psychology, sociology and neuroscience. Terms such as emulation or imitation are common in these sciences and widely discussed. The behavior transference from a human to a robot is the content of this chapter. Standing up from a chair to a stable upright posture causes conventional Zero Moment Point (ZMP) based controllers of humanoid robots to produce excessive joint torques. Humans, however, are known to manage this challenging dynamic posture control task very elegantly. This chapter proposes a novel method for humanoid robots to acquire optimal standing up behaviors based on human demonstrations. We collected 3D motion data of a group of human subjects standing up from a chair.  We solve the correspondence problem by making comparisons in a common reward space defined by a multi-objective reward function. We fitted a fully actuated triple inverted pendulum model to both human and robot motion data in order to compute a reward profile for stability and effort.  Afterwards, we used Differential Evolution optimizer to obtain a trajectory that minimizes the Kullback-Liebler divergence between the reward of the human and that of the robot, subject to constraints of ZMP, joint torques, and joint rotation limits of the robot.
This chapter presents an advancement in how a humanoid robot can learn to imitate and innovate motor skills from demonstrations of human teachers of larger kinematic structures and different actuator constraints. %We present analytical and experimental results for the task of standing up from a chair to a stable upright posture, where the robot has to transit from one stable posture to another via a set of unstable states. 
}
\newpage
%%%%%%%%%%%%%%%%%%%%%%%%%%%%%%%%%%%%%%%%%%%%%%%%%%%%%%%%%%%%%%%%%%%%%%%%%%%%%%%%
\input{Capitulo3/intro}
\input{Capitulo3/imitation_learning}
\input{Capitulo3/innovation_learning}
\input{Capitulo3/model}
\input{Capitulo3/results}
\input{Capitulo3/conclusions}
