%%%%%%%%%%%%%%%%%%%%%%%%%%%%%%%%%%%%%%%%%%%%%%%%%%%%%%%%%%%%%%%%%%%%%%%%%%%%%%%%
%2345678901234567890123456789012345678901234567890123456789012345678901234567890
%        1         2         3         4         5         6         7         8
%%%%%%%%%%%%%%%%%%%%%%%%%%%%%%%%%%%%%%%%%%%%%%%%%%%%%%%%%%%%%%%%%%%%%%%%%%%%%%%%
\documentclass[12pt,twoside]{book}

\hyphenation{self-lear-ning ro-bo-tics Te-le-co-m-mu-ni-ca-ti-ons en-vi-ron-ment par-ti-ci-pant pa-ra-me-tri-zed
Na-na-ya-kka-ra}

\usepackage[dvips]{graphicx} % for pdf, bitmapped graphics files
\usepackage{epsfig} % for postscript graphics files
%Para que acepte caracteres españoles como acentos son necesarios los tres siguiente paquetes en ese orden
\usepackage[spanish,english]{babel}
\usepackage[utf8]{inputenc}
%\usepackage[T1]{fontenc}

\usepackage{fancyhdr}
\usepackage{uc3m_thesis_fancy}
%\usepackage[nottoc,notlof,notlot]{tocbibind}
\usepackage[Lenny]{fncychap}
\usepackage{algorithm}
\usepackage{algorithmic}
\usepackage{amsmath} 
\usepackage{amssymb}
%\usepackage{algpseudocode}
\usepackage{verbatim} % comments
\usepackage{comment} % for comment blocks
\usepackage[font=small,format=plain,labelfont=bf,up,textfont=it,up]{caption}
\usepackage{anyfontsize}
%\usepackage{watermark}
\usepackage[twoside]{geometry}
\geometry{twoside, bindingoffset=1cm, papersize={210mm,297mm}, total ={150mm, 240mm}, includefoot, includehead}

\usepackage{epigraph}
\usepackage{natbib}
\def\bibpreamble{\protect\addcontentsline{toc}{chapter}{Bibliography}} %to make the blibliography appear in the TOC
%\usepackage{apalike}

\usepackage[toc]{appendix}

\setlength{\topmargin}{0cm} 
\setlength{\headsep}{8mm}
\setlength{\marginparwidth}{20mm} \setlength{\evensidemargin}{4mm} \setlength{\oddsidemargin}{10mm}

\newcommand{\clearemptydoublepage}{\newpage{\pagestyle{empty}%
\cleardoublepage}}
\newcommand{\titleofthedocument}{Humanoid Robot Control of  Complex Postural Tasks based on Learning from Demonstration} % Title of the document here
\renewcommand{\algorithmicrequire}{\textbf{Initialize:}}


\usepackage[pdftitle={\titleofthedocument}]{hyperref}
\hypersetup{colorlinks=true,linkcolor=black,urlcolor=blue,filecolor=black,citecolor=black,pdfstartview={FitH}}

\setcounter{secnumdepth}{2}%3 levels in the table of contents
\setcounter{tocdepth}{2} 

%-------------- Añadido-------------------
\setlength{\headheight}{15pt} %elimina el error de fancyhdr
\usepackage{amsmath}
\usepackage{listings}
%------------------------------------------
\graphicspath{{Capitulo1/images/}{Capitulo2/images/}{Capitulo3/images/}{Capitulo4/images/}
{Capitulo5/images/}{Capitulo6/images/}{Capitulo7/images/}{additional/images/}} 

\univ{UNIVERSIDAD CARLOS III DE MADRID}
\dept{Systems Engineering \\ and Automation}
\place{Legan\'es}


\newcommand{\robotorder}{\textit{``stand up from a chair, walk to the door and open it"}\ }
%Version control REMOVE THIS!!!!
\newcommand{\versioncontrol}[1]{\textcolor{black}{#1}}%change to black for final version, red for annotations

% Begin document
\begin{document}

\title{\titleofthedocument}

\author{Miguel González-Fierro }
\adviser{Prof. Carlos Balaguer  \\(Universidad Carlos III de Madrid)}
\coadviser{Dr. Thrishantha Nanayakkara \\(King's College London)}
\firstreader{Chair}
\secondreader{Member}
\thirdreader{Secretary}
\fourthreader{Substitute}

\datedefense{October 2014}


\beforepreface

\cleardoublepage
\newpage{\pagestyle{empty}\cleardoublepage}

%Preface
%%%%%%%%%%%%%%%%%%%%%%%%%%%%%%%%%%%%%%%%%%%%%%%%%%%%%%%%%%%%%%%%%%%%%%%%%%%%%%%%
%2345678901234567890123456789012345678901234567890123456789012345678901234567890
%        1         2         3         4         5         6         7         8
%%%%%%%%%%%%%%%%%%%%%%%%%%%%%%%%%%%%%%%%%%%%%%%%%%%%%%%%%%%%%%%%%%%%%%%%%%%%%%%%
\begin{flushright}
\emph{A mi mujer y a mi familia}
\\[12cm]
\end{flushright}
\setlength{\epigraphwidth}{11cm}
\setlength{\epigraphrule}{0pt}
\epigraph{\textit{``Well, I feel bad to lose you, but we did some nice work when we were together. Always remember that we are a lucky lot. We explore nature looking for her secret laws that govern our lives. [...] Our thirst is to uncover the secrets. That is exciting. Each finding we make changes the way people feel about their world. You must continue that. The human brain is an amazing thing. [...] Our brains are designed to  think. They are thinking machines. The more you imagine, the more you enjoy. Never be followers. They are next to dead bodies. All I want you to do is to continue challenging life.''}}{--- \textup{Thrishantha Nanayakkara in \textit{Devi}}}


%%%%%%%%%%%%%%%%%%%%%%%%%%%%%%%%%%%%%%%%%%%%%%%%%%%%%%%%%%%%%%%%%%%%%%%%%%%%%%%%
%\newpage
%\textbf{The illustrated guide to a PhD}\\

%Imagine a circle that contains all of human knowledge:

%TODO: pasar a vectorial
%\begin{center}
%	\includegraphics[scale=0.2]{PhDKnowledge001.jpg}
%\end{center}
%By the time you finish elementary school, you know a little:
%\begin{center}
%	\includegraphics[scale=0.2]{PhDKnowledge002.jpg}
%\end{center}
%By the time you finish high school, you know a bit more:
%\begin{center}
%	\includegraphics[scale=0.2]{PhDKnowledge003.jpg}
%\end{center}
%With a bachelor's degree, you gain a specialty:
%\begin{center}
%	\includegraphics[scale=0.2]{PhDKnowledge004.jpg}
%\end{center}
%A master's degree deepens that specialty:
%\begin{center}
%	\includegraphics[scale=0.2]{PhDKnowledge005.jpg}
%\end{center}
%Reading research papers takes you to the edge of human knowledge:
%\begin{center}
%	\includegraphics[scale=0.2]{PhDKnowledge006.jpg}
%\end{center}
%Once you're at the boundary, you focus:
%\begin{center}
%	\includegraphics[scale=0.2]{PhDKnowledge007.jpg}
%\end{center}
%You push at the boundary for a few years:
%\begin{center}
%	\includegraphics[scale=0.2]{PhDKnowledge008.jpg}
%\end{center}
%Until one day, the boundary gives way:
%\begin{center}
%	\includegraphics[scale=0.2]{PhDKnowledge009.jpg}
%\end{center}
%And, that dent you've made is called a Ph.D.:
%\begin{center}
%	\includegraphics[scale=0.2]{PhDKnowledge010.jpg}
%\end{center}
%Of course, the world looks different to you now:
%\begin{center}
%	\includegraphics[scale=0.2]{PhDKnowledge011.jpg}
%\end{center}
%So, don't forget the bigger picture:
%\begin{center}
%	\includegraphics[scale=0.2]{PhDKnowledge012.jpg}
%\end{center}
%Keep pushing \citep{mightphd}.










\clearemptydoublepage

\prefacesection{Acknowledgements}
%%%%%%%%%%%%%%%%%%%%%%%%%%%%%%%%%%%%%%%%%%%%%%%%%%%%%%%%%%%%%%%%%%%%%%%%%%%%%%%%
%2345678901234567890123456789012345678901234567890123456789012345678901234567890
%        1         2         3         4         5         6         7         8
%%%%%%%%%%%%%%%%%%%%%%%%%%%%%%%%%%%%%%%%%%%%%%%%%%%%%%%%%%%%%%%%%%%%%%%%%%%%%%%%
I started a Ph.D. because when I finished my degree I felt that I had learned very few things about Engineering. It is funny that now that I  finished the thesis, I feel kind of the same way. Even worse, the more I study, the more I understand that I have many lessons to learn. I realised that there are two kinds of  people, those who don't know and those who don't know they don't know. Those who study all their life and those who think they know everything. When you are a scientist, you surely are  one of the students. 

I am a privileged person, I am very lucky. I was granted with the opportunity to put a brick in the wall of science. I stood on the shoulders of giants as many people did before me.  It is even more exciting to participate in the growing of robotics, a very young science in comparison with others like physics or mathematics. In the next decades there will be robots collaborating with us and this future will have a piece of my work and effort. 

I would like to express my deepest gratitude to everyone that have accompanied me in all this years of learning. Therefore, I would like to propose a toast:

To Carlos Balaguer. For his leadership and kindness. For giving me the opportunity to learn at his side. For all the time he has given me, which is more valuable knowing he does not have much. And for his guidance and help in all the  decisions I have made. 

To Thrishantha Nanayakkara. For accepting me as his Ph.D student and giving me the opportunity of being part of his research group in London. For everything that I have learned from him and all the things I  will learn. For his wisdom and knowledge. And for being so close even though we were so far. 

To Luis Moreno and Santiago Garrido. For being my  Ph.D. advisors in the shadows. For all that have learned from them and for answering all my questions. 

To Juan Gonz{\'a}lez V{\'i}ctores, Daniel Hern{\'a}ndez, Martin Stoelen and V{\'i}ctor Gonz{\'a}lez Pacheco. For all the ideas and experiences we have shared.

To Jorge Garc{\'i}a Bueno and Alejandro Mart{\'i}n. For being my partners in the difficult adventure of creating a company.

To  Concha Monje, Dolores Blanco, Paolo Pierro, Alberto Jard{\'o}n, Javier Gorostiza, Ram{\'on} Barber and Santiago Mart{\'i}nez de la Casa. For  helping me every time I have needed them and for all the funny moments we have lived together. 

To Raúl Perula, Juanmi García Haro, Santi Morante, Javier Quijano and  every member of RoboticsLab. For all the jokes and the good experiences.

To Sonia Mata and Eduardo Silles. For solving many problems and being always ready to help.

To Juan Dom\'inguez. For being the one to encourage me to study Industrial Engineer.

To David L{\'o}pez del Moral, Mª Jes{\'u}s G{\'o}mez and Eduardo Corral. For being as crazy as I am to do a Ph.D. and for all the afternoons we have spent together.

To Miguel Maldonado. For being my brother. For supporting me, working with me and helping me. And for being at my side so many time.

To my father, sister, grandparents and the rest of my family. For their support and love.

To my mother. For being the best mother anyone can desire. For his wisdom and dedication.

To my wife. For being everything to me.

\vspace{20mm}
\begin{flushright}
\textit{Gracias a todos}
\end{flushright}
\begin{flushright}
\textit{Miguel}
\end{flushright}






\clearemptydoublepage

\prefacesection{Abstract}
%%%%%%%%%%%%%%%%%%%%%%%%%%%%%%%%%%%%%%%%%%%%%%%%%%%%%%%%%%%%%%%%%%%%%%%%%%%%%%%%
%2345678901234567890123456789012345678901234567890123456789012345678901234567890
%        1         2         3         4         5         6         7         8
%%%%%%%%%%%%%%%%%%%%%%%%%%%%%%%%%%%%%%%%%%%%%%%%%%%%%%%%%%%%%%%%%%%%%%%%%%%%%%%%
This thesis addresses the problem of planning and controlling complex tasks in a humanoid robot from a postural point of view. It is motivated by the growth of robotics in our current society, where simple robots are being integrated. Its objective is to make an advancement in the development of complex behaviors in humanoid robots, in order to allow them to share our environment in the future. 

The work presents different contributions in the areas of humanoid robot postural control, behavior planning, non-linear control, learning from demonstration and reinforcement learning. First, as an introduction of the thesis, a group of methods and mathematical formulations are presented, describing concepts such as humanoid robot modelling, generation of locomotion trajectories and generation of whole-body trajectories.
 
Next, the process of human learning is studied in order to develop a novel method of postural task transference between a human and a robot. It uses  the demonstrated action goal as a metrics of comparison, which is codified using the reward associated to the task execution. 

As an evolution of the previous study, this process is generalized to a set of sequential behaviors, which are executed by the robot based on human demonstrations. 

Afterwards, the execution of postural movements using a robust control  approach is proposed. This method allows to control the desired trajectory even with mismatches in the robot model. 

Finally, an architecture that encompasses all  methods of postural planning and control is presented. It is complemented by an environment recognition module that identifies the free space in order to perform path planning and generate safe movements for the robot. 

The experimental justification of this thesis was developed using the humanoid robot HOAP-3. Tasks such as walking, standing up from a chair, dancing or opening a door have been implemented using the techniques proposed in this work. 



\clearemptydoublepage

\prefacesection{Resumen}
%%%%%%%%%%%%%%%%%%%%%%%%%%%%%%%%%%%%%%%%%%%%%%%%%%%%%%%%%%%%%%%%%%%%%%%%%%%%%%%%
%2345678901234567890123456789012345678901234567890123456789012345678901234567890
%        1         2         3         4         5         6         7         8
%%%%%%%%%%%%%%%%%%%%%%%%%%%%%%%%%%%%%%%%%%%%%%%%%%%%%%%%%%%%%%%%%%%%%%%%%%%%%%%%
%\vspace{-2cm}
Esta tesis aborda el problema de la planificación y control de tareas complejas de un robot humanoide desde el punto de vista postural. Viene motivada por el auge  de la robótica en la sociedad actual, donde ya se están incorporando robots sencillos y su objetivo es avanzar en el desarrollo de comportamientos complejos en robots humanoides, para que en el futuro sean capaces de compartir nuestro entorno.

El trabajo presenta diferentes contribuciones en las áreas de control postural de robots humanoides, planificación de comportamientos, control no lineal, aprendizaje por demostración y aprendizaje por refuerzo. En primer lugar se desarrollan un conjunto de métodos y formulaciones matemáticas sobre los que se sustenta la tesis, describiendo conceptos de modelado de robots humanoides, generación de trayectorias de locomoción y generación de trayectorias del cuerpo completo. 

A continuación se estudia el proceso de aprendizaje humano, para desarrollar un novedoso método de transferencia de una tarea postural de un humano a un robot, usando como métrica de comparación el objetivo de la acción demostrada, que es codificada a través del refuerzo asociado a la ejecución de dicha tarea. 

Como evolución del trabajo anterior, se generaliza este proceso para la realización de un conjunto de comportamientos secuenciales, que son de nuevo realizados por el robot basándose en las demostraciones de un ser humano.

Seguidamente se estudia la ejecución de movimientos posturales utilizando un método de control robusto ante imprecisiones en el modelado del robot. 

Para finalizar, se presenta una arquitectura que aglutina los métodos de planificación y el control postural desarrollados en los capítulos anteriores. Esto se complementa con un módulo de reconocimiento del entorno y extracción del espacio libre para poder planificar y generar movimientos seguros en dicho entorno.

La justificación experimental de la tesis se ha desarrollado con el robot humanoide HOAP-3. En este robot se han implementado tareas como caminar, levantarse de una silla, bailar o abrir una puerta. Todo ello haciendo uso de las técnicas propuestas en este trabajo.


%



\clearemptydoublepage

% Igual necesito una sección de abreviaciones
%\prefacesection{Abbreviations}
%\hspace{5 mm} 2D - two-dimensional



\afterpreface
\pagenumbering{arabic}
\pagestyle{fancyplain}
\renewcommand{\chaptermark}[1] %
{\markboth{#1}{\thechapter\ #1}}
\renewcommand{\sectionmark}[1]%
{\markright{\thesection\ #1}}
\lhead[\fancyplain{}{\bfseries\thepage}]
{\fancyplain{}{\bfseries\rightmark}}
\rhead[\fancyplain{}{\bfseries\leftmark}] {\fancyplain{}{\bfseries\thepage}}
\cfoot{}

%------------------------------------------	
%%%%%%%%%%%%%%%%%%%%%%%%%%%%%%%%%%%%%%%%%%%%%%%%%%%%%%%%%%%%%%%%%%%%%%%%%%%%%%%%
%2345678901234567890123456789012345678901234567890123456789012345678901234567890
%        1         2         3         4         5         6         7         8
%%%%%%%%%%%%%%%%%%%%%%%%%%%%%%%%%%%%%%%%%%%%%%%%%%%%%%%%%%%%%%%%%%%%%%%%%%%%%%%%
\chapter{Introduction} \label{ch_intro}
\textit{This chapter deals with the initial introduction, motivation and presentation  of this thesis. The thesis addresses aspects related to planning and control of complex postural tasks for humanoid robots using human learning by demonstration. The thesis proposes a specific situation that has to be solved by the robot. It is the order \robotorder that represents a complex task which involves the development of several skills and gives rise to a broad set of approaches to solve this problem. The present work is focused on discussing and studying methods that involves the generation and execution of humanoid motions in terms of postural body transitions. These postural body transitions are optimized using an index that we called the reward profile. The reward profile  is a multi-modal time-dependent function that encodes the skill goal that the robot has to perform. At the same time, it is a measurement of the skill performance in terms of different aspects like stability, softness or human likeliness. Furthermore, this thesis addresses the modelling and control of the humanoid robot and presents a wide variety of simulated and experimental results. The objective behind this work is to make humanoid robots more intelligent and autonomous, by allowing them to follow complex orders safely and precisely.}
\newpage
%%%%%%%%%%%%%%%%%%%%%%%%%%%%%%%%%%%%%%%%%%%%%%%%%%%%%%%%%%%%%%%%%%%%%%%%%%%%%%%%
%\input{Capitulo1/motivation}
%\input{Capitulo1/objectives}
%\input{Capitulo1/organization}
%%%%%%%%%%%%%%%%%%%%%%%%%%%%%%%%%%%%%%%%%%%%%%%%%%%%%%%%%%%%%%%%%%%%%%%%%%%%%%%%









\clearemptydoublepage
%------------------------------------------	
%%%%%%%%%%%%%%%%%%%%%%%%%%%%%%%%%%%%%%%%%%%%%%%%%%%%%%%%%%%%%%%%%%%%%%%%%%%%%%%%
%2345678901234567890123456789012345678901234567890123456789012345678901234567890
%        1         2         3         4         5         6         7         8
\chapter{Basic Representations for Postural Control in Humanoids}\label{ch_basics}
\textit{This chapter deals with some basic concepts that are repeatably used in this thesis. There are some basic tasks that have to be done in advance in order to conduct many of the experiments. They are related to modeling, kinematics, control and dynamics. There are many aspects of the humanoid set up that are not explicitly described in the thesis. This chapter acts as the basis of some of the algorithms that are explained in the next chapters. First, a study of different humanoid robot models is presented. It includes simple models to represent the robot, like the inverted pendulum, and complex models like the mass distributed model. The equation of motion of each of them is obtained. These models are included in some  motion generation algorithms and balance maintenance methods. Afterwards, the most famous stability criterion, the Zero Moment Point (ZMP),  is explained in detail. Furthermore, two common methods of biped locomotion generation are studied, the 3D Linear Inverted Pendulum Model and the Cart Table model. The latter was used to generate stable biped locomotion in the real humanoid. Finally, a simple method of whole body imitation is presented and a dance performance is obtained as an experiment. }
\newpage
%%%%%%%%%%%%%%%%%%%%%%%%%%%%%%%%%%%%%%%%%%%%%%%%%%%%%%%%%%%%%%%%%%%%%%%%%%%%%%%%
\input{Capitulo2/intro}
\input{Capitulo2/model}
\input{Capitulo2/generation}
\input{Capitulo2/dancing}
\input{Capitulo2/conclusions}

\clearemptydoublepage
%------------------------------------------	
%%%%%%%%%%%%%%%%%%%%%%%%%%%%%%%%%%%%%%%%%%%%%%%%%%%%%%%%%%%%%%%%%%%%%%%%%%%%%%%%
%2345678901234567890123456789012345678901234567890123456789012345678901234567890
%        1         2         3         4         5         6         7         8
\chapter{Imitation Learning and Skill Innovation in Humanoids through Reward Templates}\label{ch_imitation}
\textit{The imitation of a skill and its improvement by innovating new solutions are  the key steps to human learning. The process of human learning  has been widely studied in psychology, sociology and neuroscience. Terms such as emulation or imitation are common in these sciences and widely discussed. The behavior transference from a human to a robot is the content of this chapter. Standing up from a chair to a stable upright posture causes conventional Zero Moment Point (ZMP) based controllers of humanoid robots to produce excessive joint torques. Humans, however, are known to manage this challenging dynamic posture control task very elegantly. This chapter proposes a novel method for humanoid robots to acquire optimal standing up behaviors based on human demonstrations. We collected 3D motion data of a group of human subjects standing up from a chair.  We solve the correspondence problem by making comparisons in a common reward space defined by a multi-objective reward function. We fitted a fully actuated triple inverted pendulum model to both human and robot motion data in order to compute a reward profile for stability and effort.  Afterwards, we used Differential Evolution optimizer to obtain a trajectory that minimizes the Kullback-Liebler divergence between the reward of the human and that of the robot, subject to constraints of ZMP, joint torques, and joint rotation limits of the robot.
This chapter presents an advancement in how a humanoid robot can learn to imitate and innovate motor skills from demonstrations of human teachers of larger kinematic structures and different actuator constraints. %We present analytical and experimental results for the task of standing up from a chair to a stable upright posture, where the robot has to transit from one stable posture to another via a set of unstable states. 
}
\newpage
%%%%%%%%%%%%%%%%%%%%%%%%%%%%%%%%%%%%%%%%%%%%%%%%%%%%%%%%%%%%%%%%%%%%%%%%%%%%%%%%
\input{Capitulo3/intro}
\input{Capitulo3/imitation_learning}
\input{Capitulo3/innovation_learning}
\input{Capitulo3/model}
\input{Capitulo3/results}
\input{Capitulo3/conclusions}

\clearemptydoublepage
%------------------------------------------
%%%%%%%%%%%%%%%%%%%%%%%%%%%%%%%%%%%%%%%%%%%%%%%%%%%%%%%%%%%%%%%%%%%%%%%%%%%%%%%%
%2345678901234567890123456789012345678901234567890123456789012345678901234567890
%        1         2         3         4         5         6         7         8
%%%%%%%%%%%%%%%%%%%%%%%%%%%%%%%%%%%%%%%%%%%%%%%%%%%%%%%%%%%%%%%%%%%%%%%%%%%%%%%%
\chapter{Learning and Improving a Sequence of Goal Directed Skills}\label{ch_multiple_behaviors}
\textit{As it was discussed in the previous chapter, there are evidences that justify that the imitation between humans are goal-directed. We proposed there a new method to acquire a single skill from human demonstrations. However, it is quite common for a human being to perform several skills sequentially, for example, to walk to a door and open it. Therefore, when performing multiple skills, we internally define an unknown optimal policy to satisfy multiple goals. This chapter presents a method to transfer a complex behavior composed by multiple skills  from a human demonstrator to a humanoid robot. We defined a multi-objective reward function as a measurement of the goal optimality for both human and robot, which is defined in each subtask of the global behavior. We optimized a hierarchical policy to generate whole-body movements for the robot that produces a reward profile that is compared and matched with the human reward profile, producing an imitative behavior. Furthermore, we can search in the proximity of the solution space to improve the reward profile and innovate a new solution, which is more beneficial for the humanoid. Experiments were carried out in a real  humanoid robot.}
\newpage
%%%%%%%%%%%%%%%%%%%%%%%%%%%%%%%%%%%%%%%%%%%%%%%%%%%%%%%%%%%%%%%%%%%%%%%%%%%%%%%%
% chapter1
\input{Capitulo4/intro}
\input{Capitulo4/related_work}
\input{Capitulo4/open_door}
%chapter2
\input{Capitulo4/policy}
\input{Capitulo4/gmm_gmr}
\input{Capitulo4/primitives}
\input{Capitulo4/policy_optimization}
%chapter3
\input{Capitulo4/demonstrations}
\input{Capitulo4/behavior_selector}
\input{Capitulo4/reward}
\input{Capitulo4/trajectory_generation}
%chapter4
\input{Capitulo4/conclusions}

\clearemptydoublepage
%------------------------------------------
%%%%%%%%%%%%%%%%%%%%%%%%%%%%%%%%%%%%%%%%%%%%%%%%%%%%%%%%%%%%%%%%%%%%%%%%%%%%%%%%
%2345678901234567890123456789012345678901234567890123456789012345678901234567890
%        1         2         3         4         5         6         7         8
%%%%%%%%%%%%%%%%%%%%%%%%%%%%%%%%%%%%%%%%%%%%%%%%%%%%%%%%%%%%%%%%%%%%%%%%%%%%%%%%
\chapter{Robust Control of Humanoid Models through Fractional Calculus}\label{frac_chapter}
\textit{There is an open discussion between those who defend mass distributed models for humanoid robots and those in favor of simple concentrated models. Even though each of them has its advantages and disadvantages, little research has been conducted analyzing the control performance due to the mismatch between the model and the real robot, and how the simplifications affect the controller’s output. In this chapter we address this problem by combining a reduced model of the humanoid robot, which has an easy mathematical formulation and implementation, with a fractional order controller, which is robust to changes in the model parameters. This controller is a generalization of the well-known PID structure obtained from the application of Fractional Calculus to control. This control strategy guarantees the robustness of the system, minimizing the effects from the assumption that the robot has a simple mass distribution. The humanoid robot is modeled and identified as a triple inverted pendulum and, using a gain scheduling strategy, the performances of a classical PID controller and a fractional order PID controller are compared, tuning the controller parameters with a genetic algorithm.}
\newpage
%%%%%%%%%%%%%%%%%%%%%%%%%%%%%%%%%%%%%%%%%%%%%%%%%%%%%%%%%%%%%%%%%%%%%%%%%%%%%%%%
\input{Capitulo5/intro}
\input{Capitulo5/related_work}
\input{Capitulo5/fractional_controlers}
\input{Capitulo5/humanoid_robust_control}
\input{Capitulo5/results}
\input{Capitulo5/conclusions}




\clearemptydoublepage
%------------------------------------------
%%%%%%%%%%%%%%%%%%%%%%%%%%%%%%%%%%%%%%%%%%%%%%%%%%%%%%%%%%%%%%%%%%%%%%%%%%%%%%%%
%2345678901234567890123456789012345678901234567890123456789012345678901234567890
%        1         2         3         4         5         6         7         8
%%%%%%%%%%%%%%%%%%%%%%%%%%%%%%%%%%%%%%%%%%%%%%%%%%%%%%%%%%%%%%%%%%%%%%%%%%%%%%%%
%\chapter{Postural Planning and Control in Complex Environments}\label{ch_postural_control}
\chapter{Control of Humanoid Robots Executing Complex Tasks}\label{ch_postural_control}
\textit{This chapter deals with the planning and execution of a complex task ordered to a humanoid robot. The robot has to be able to execute high level postural tasks in the presence of a cluttered environment. The motion execution must be soft and stable and, at the same time, the robot has to be able to successfully avoid obstacles in the environment. First, the robot has to identify the environment and the obstacles. Second, the robot has to be able to move from the initial point to the final point performing a set of postural movements. The postural task is performed in two levels, a postural planning, which off-line computes  the  safe and stable postural movement that allows the robot to  navigate through the environment, and an online postural control, which has to do with the execution, control and disturbance rejection that makes possible the fulfillment of the task. This chapter encompasses the learning strategies explained in chapters \ref{ch_imitation} and \ref{ch_multiple_behaviors}, the control method of chapter \ref{frac_chapter}, while using methodologies of chapter \ref{ch_basics}. The task selected as an example is a robot that starts seated on a chair, stands up, walks avoiding obstacles until it reaches a door, opens the door and leaves the room. This chapter also gives a practical significance to this thesis. }
\newpage
%%%%%%%%%%%%%%%%%%%%%%%%%%%%%%%%%%%%%%%%%%%%%%%%%%%%%%%%%%%%%%%%%%%%%%%%%%%%%%%%

%\input{Capitulo6/intro}
%\input{Capitulo6/environment}
%\input{Capitulo6/postural_planning}
%\input{Capitulo6/postural_control}
%\input{Capitulo6/results}
%\input{Capitulo6/conclusions}

\clearemptydoublepage
%------------------------------------------
%%%%%%%%%%%%%%%%%%%%%%%%%%%%%%%%%%%%%%%%%%%%%%%%%%%%%%%%%%%%%%%%%%%%%%%%%%%%%%%%
%2345678901234567890123456789012345678901234567890123456789012345678901234567890
%        1         2         3         4         5         6         7         8
%%%%%%%%%%%%%%%%%%%%%%%%%%%%%%%%%%%%%%%%%%%%%%%%%%%%%%%%%%%%%%%%%%%%%%%%%%%%%%%%
\chapter{Conclusions and Future Works}\label{ch_conclusions}
\textit{ This thesis has attempted to contribute in some areas related to  posture behavior of humanoid robots.  From the beginning, this work has had a differential feature if it is compared with a usual Ph.D. thesis. It has an initial framework which is the center of all  developed studies. The framework is the high level order  \robotorder that the robot needs to acomplish. To execute that order, the humanoid robot needs to perform a set of  movements, it needs to avoid a series of obstacles and it needs to adopt a determined posture or set of postures. All needed procedures to successfully achieve this order are gathered in this thesis. The amount of contributions that this thesis has generated are related to learning from demonstration, reinforcement learning, non-linear control and motion planning. In this chapter the conclusions of this thesis are summarized and a discussion of the positive  as well as the negative aspects of the developed work is presented. There are also proposed some future lines of development that can be used to improve the current work or to serve as a start point for new developments. }
\newpage
%%%%%%%%%%%%%%%%%%%%%%%%%%%%%%%%%%%%%%%%%%%%%%%%%%%%%%%%%%%%%%%%%%%%%%%%%%%%%%%%
%\input{Capitulo7/conclusions}
%\input{Capitulo7/contributions}
%\input{Capitulo7/future}
%\input{Capitulo7/publications}







\clearemptydoublepage
%------------------------------------------
\appendix
\input{./additional/hoap_humanoid}
\clearemptydoublepage
\input{./additional/appendix_human}
\clearemptydoublepage

%------------------------------
% Bibliografía
%------------------------------------------

%\addcontentsline{toc}{chapter}{Bibliography}
\bibliography{ref}
%\bibliographystyle{apacite}
\bibliographystyle{apalike}

\end{document}
