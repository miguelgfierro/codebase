%%%%%%%%%%%%%%%%%%%%%%%%%%%%%%%%%%%%%%%%%%%%%%%%%%%%%%%%%%%%%%%%%%%%%%%%%%%%%%%%
%2345678901234567890123456789012345678901234567890123456789012345678901234567890
%        1         2         3         4         5         6         7         8
%%%%%%%%%%%%%%%%%%%%%%%%%%%%%%%%%%%%%%%%%%%%%%%%%%%%%%%%%%%%%%%%%%%%%%%%%%%%%%%%
This thesis addresses the problem of planning and controlling complex tasks in a humanoid robot from a postural point of view. It is motivated by the growth of robotics in our current society, where simple robots are being integrated. Its objective is to make an advancement in the development of complex behaviors in humanoid robots, in order to allow them to share our environment in the future. 

The work presents different contributions in the areas of humanoid robot postural control, behavior planning, non-linear control, learning from demonstration and reinforcement learning. First, as an introduction of the thesis, a group of methods and mathematical formulations are presented, describing concepts such as humanoid robot modelling, generation of locomotion trajectories and generation of whole-body trajectories.
 
Next, the process of human learning is studied in order to develop a novel method of postural task transference between a human and a robot. It uses  the demonstrated action goal as a metrics of comparison, which is codified using the reward associated to the task execution. 

As an evolution of the previous study, this process is generalized to a set of sequential behaviors, which are executed by the robot based on human demonstrations. 

Afterwards, the execution of postural movements using a robust control  approach is proposed. This method allows to control the desired trajectory even with mismatches in the robot model. 

Finally, an architecture that encompasses all  methods of postural planning and control is presented. It is complemented by an environment recognition module that identifies the free space in order to perform path planning and generate safe movements for the robot. 

The experimental justification of this thesis was developed using the humanoid robot HOAP-3. Tasks such as walking, standing up from a chair, dancing or opening a door have been implemented using the techniques proposed in this work. 


